%CS-113 S18 HW-2
%Released: 2-Feb-2018
%Deadline: 16-Feb-2018 7.00 pm
%Authors: Abdullah Zafar, Emad bin Abid, Moonis Rashid, Abdul Rafay Mehboob, Waqar Saleem.


\documentclass[addpoints]{exam}

% Header and footer.
\pagestyle{headandfoot}
\runningheadrule
\runningfootrule
\runningheader{CS 113 Discrete Mathematics}{Homework II}{Spring 2018}
\runningfooter{}{Page \thepage\ of \numpages}{}
\firstpageheader{}{}{}

\boxedpoints
\printanswers
\usepackage[table]{xcolor}
\usepackage{amsfonts,graphicx,amsmath,hyperref}
\title{Habib University\\CS-113 Discrete Mathematics\\Spring 2018\\HW 2}
\author{$lk04067$}  % replace with your ID, e.g. oy02945
\date{Due: 19h, 16th February, 2018}


\begin{document}
\maketitle

\begin{questions}



\question

%Short Questions (25)

\begin{parts}

 
  \part[5] Determine the domain, codomain and set of values for the following function to be 
  \begin{subparts}
  \subpart Partial
  \subpart Total
  \end{subparts}

  \begin{center}
    $y=\sqrt{x}$
  \end{center}

  \begin{solution}
    % Write your solution here
    
    Let,\\ 
    $\mathbb{R}$ = Set of Real numbers,\\
    $\mathbb{C}$ = Set of Complex numbers \\
    Also, let S be the set of values for the function.\\ 
    
    \begin{subparts}
    \subpart \textbf{Partial function}\\
        Domain: $\mathbb{R}$   \\
        Codomain: $\mathbb{R^+}$\\
        Set of Values: all non-negative integers\\
        
    \subpart \textbf{Total function}\\
        Domain: $\mathbb{R}$\\
        Codomain: $\mathbb{C}$\\
        Set of Values: $\mathbb{R}$ \\
    \end{subparts}
    
  \end{solution}
  
  \part[5] Explain whether $f$ is a function from the set of all bit strings to the set of integers if $f(S)$ is the smallest $i \in \mathbb{Z}$ such that the $i$th bit of S is 1 and $f(S) = 0$ when S is the empty string. 
  
  \begin{solution}
    % Write your solution here
    According to the function definition, f(S) is the smallest bit position ($i$) of 1 in a bit string (S). For an empty string, the function value is defined to be zero.\\
    
    Consider a bit string of n bits such that all bits are zero. For example S= 00, S= 000, S= 0000, ... etc.\\
    
    In the above examples, no bit is 1, it is impossible to find the position of 1 (i.e. the function value).\\
    
    Since f(S) is not well-defined for a string of all 0s, f(S) is not a function. \\
    
  \end{solution}

  \part[15] For $X,Y \in S$, explain why (or why not) the following define an equivalence relation on $S$:
  \begin{subparts}
    \subpart ``$X$ and $Y$ have been in class together"
    \subpart ``$X$ and $Y$ rhyme"
    \subpart ``$X$ is a subset of $Y$"
  \end{subparts}

  \begin{solution}
    % Write your solution here
    Equivalence relation is defined as a relation between the elements of a set which is reflexive, symmetric, and transitive.
    
    \begin{subparts}
    \subpart \textbf{``$X$ and $Y$ have been in class together"}\\
    Transitive property does not hold in this relation.\\
    Here, the class is not a certain specified class. So, X can be in a class A with Z while Y being in any other class B with Z. And this cannot be inferred as X and Y have been in class together (unless A=B).\\
    Since the transitive property is not present, ``$X$ and $Y$ have been in class together" is not an equivalence relation on S.\\
    \subpart \textbf{``$X$ and $Y$ rhyme"}\\
    
    a- Reflexive property:\\
    $\forall$ X $\forall$ Y (``$X$ and $X$ rhyme" $\wedge$ ``$Y$ and $Y$ rhyme") = True\\
    Reflexive property holds.\\
    
    b- Symmetric Property:\\
     $\forall$ X $\forall$ Y(``$X$ and $Y$ rhyme" $\Leftrightarrow$ ``$Y$ and $X$ rhyme")\\
     The relation is symmetric.\\
    
    c- Transitive Property:\\
    $\forall$ X $\forall$ Y $\forall$ Z (``$X$ and $Z$ rhyme" $\wedge$ ``$Z$ and $Y$ rhyme" $\rightarrow$ ``$X$ and $Y$ rhyme")
    Transitive property holds.\\
    
    ``$X$ and $Y$ rhyme" is an equivalence relation on S.
    
    
    \subpart \textbf{``$X$ is a subset of $Y$"}
    This relation is not symmetric.\\
    ``$X$ is a subset of $Y$" does not imply that ``$Y$ is a subset of $X$ unless X=Y\\
    For example:\\
    Let X= $\{1,2,3\}$ and Y= $\{1,2,3,4\}$,
    here,\\
    X $\subset$ Y but Y $\not\subset$ X\\
    
    ``$X$ is a subset of $Y$" is not an equivalence relation on S.
    
    
    
  \end{subparts}
  \end{solution}

\end{parts}

%Long questions (75)
\question[15] Let $A = f^{-1}(B)$. Prove that $f(A) \subseteq B$.
  \begin{solution}\\
    % Write your solution here
    The inverse function of $f$ is the function that assigns to an element b belonging to B the unique element a in A such that $f(a) = b$. The inverse function of $f$ is denoted by $f^{-1}$. Hence, $f^{-1}(b)$ = a when $f(a) = b$.\\
    
    Let a $\in$ A and b $\in$ B.
    
    Since  $A = f^{-1}(B)$ --(i)\\
    
    $\Rightarrow a \in f^{-1}(B)$
    
    Using the definition of inverse function,\\
    $b \in f(f^{-1}(B)) $
    
    $(i) \Rightarrow b \in f(A)$\\
    
    then,\\
    
    $\Rightarrow \forall$ b [$(b \in f(A))\wedge$ (b $\in$ B) $\rightarrow$  $f(A) = B $]\\
    
    $\Rightarrow f(A) \subseteq B  $
    
    
    
    
  \end{solution}

\question[15] Consider $[n] = \{1,2,3,...,n\}$ where $n \in \mathbb{N}$. Let $A$ be the set of subsets of $[n]$ that have even size, and let $B$ be the set of subsets of $[n]$ that have odd size. Establish a bijection from $A$ to $B$, thereby proving $|A| = |B|$. (Such a bijection is suggested below for $n = 3$) 
 
\begin{center}

  \begin{tabular}{ |c || c | c | c |c |}
    \hline
 A & $\emptyset$ & $\{2,3\}$ & $\{1,3\}$ & $\{1,2\}$ \\ \hline
 B & $\{3\}$ & $\{2\}$ & $\{1\}$ & $\{1,2,3\}$\\\hline
\end{tabular}
\end{center}

  \begin{solution}
    % Write your solution here
    Considering the given example, we can establish a bijection from A to B as follows:\\
\[    
    B= \forall S \in A
    \begin{cases}
         (\textrm{S $\cup \{$n$\}$ } &  \textrm{if n $\notin$ S}\\
         (\textrm { S - $\{$n$\}$ } &  \textrm{if n $\in$ S}
  
    \end{cases}
\]

    Since there is a bijection between A and B, we know that for every element in A there is a corresponding element in B (and vice versa). This, hence, proves that $|A|= |B|$
  \end{solution}
  
\question Mushrooms play a vital role in the biosphere of our planet. They also have recreational uses, such as in understanding the mathematical series below. A mushroom number, $M_n$, is a figurate number that can be represented in the form of a mushroom shaped grid of points, such that the number of points is the mushroom number. A mushroom consists of a stem and cap, while its height is the combined height of the two parts. Here is $M_5=23$:

\begin{figure}[h]
  \centering
  \includegraphics[scale=1.0]{m5_figurate.png}
  \caption{Representation of $M_5$ mushroom}
  \label{fig:mushroom_anatomy}
\end{figure}

We can draw the mushroom that represents $M_{n+1}$ recursively, for $n \geq 1$:
\[ 
    M_{n+1}=
    \begin{cases} 
      f(\textrm{Cap\_width}(M_n) + 1, \textrm{Stem\_height}(M_n) + 1, \textrm{Cap\_height}(M_n))  & n \textrm{ is even} \\
      f(\textrm{Cap\_width}(M_n) + 1, \textrm{Stem\_height}(M_n) + 1, \textrm{Cap\_height}(M_n)+1) & n \textrm{ is odd}  \\      
   \end{cases}
\]

Study the first five mushrooms carefully and make sure you can draw subsequent ones using the recurrence above.

\begin{figure}[h]
  \centering
  \includegraphics{mushroom_series.png}
  \caption{Representation of $M_1,M_2,M_3,M_4,M_5$ mushrooms}
  \label{fig:mushroom_anatomy}
\end{figure}

  \begin{parts}
    \part[15] Derive a closed-form for $M_n$ in terms of $n$.
  \begin{solution}\\
    % Write your solution here
    The patterns deduced from the provided recurrence:\\
    
    Cap Width=n+1\\
    
    Stem Height=n-1 \\
    
    Cap Height= $\lceil \frac{n+1}{2}\rceil$\\
    
    The sum of Arithmetic Progression is\\
    
    $S_{n}=\frac{n}{2}[2a+(n-1)d]$ \\
    
    $M_{n} = \frac{Cap Height}{2}[2(Cap Width)-(Cap Height-1)] + 2(Stem Height)$\\
    
    $M_{n} = \frac{\lceil \frac{n+1}{2}\rceil}{2}[2(n+1)-(\lceil \frac{n+1}{2}\rceil-1)] + 2(n-1)$\\
    
    
    
  \end{solution}
    \part[5] What is the total height of the $20$th mushroom in the series? 
  \begin{solution}
    % Write your solution here
    Total height of 20th mushroom= 30
  \end{solution}
\end{parts}

\question
    The \href{https://en.wikipedia.org/wiki/Fibonacci_number}{Fibonacci series} is an infinite sequence of integers, starting with $1$ and $2$ and defined recursively after that, for the $n$th term in the array, as $F(n) = F(n-1) + F(n-2)$. In this problem, we will count an interesting set derived from the Fibonacci recurrence.
    
The \href{http://www.maths.surrey.ac.uk/hosted-sites/R.Knott/Fibonacci/fibGen.html#section6.2}{Wythoff array} is an infinite 2D-array of integers where the $n$th row is formed from the Fibonnaci recurrence using starting numbers $n$ and $\left \lfloor{\phi\cdot (n+1)}\right \rfloor$ where $n \in \mathbb{N}$ and $\phi$ is the \href{https://en.wikipedia.org/wiki/Golden_ratio}{golden ratio} $1.618$ (3 sf).

\begin{center}
\begin{tabular}{c c c c c c c c}
 \cellcolor{blue!25}1 & 2 & 3 & 5 & 8 & 13 & 21 & $\cdots$\\
 4 & \cellcolor{blue!25}7 & 11 & 18 & 29 & 47 & 76 & $\cdots$\\
 6 & 10 & \cellcolor{blue!25}16 & 26 & 42 & 68 & 110 & $\cdots$\\
 9 & 15 & 24 & \cellcolor{blue!25}39 & 63 & 102 & 165 & $\cdots$ \\
 12 & 20 & 32 & 52 & \cellcolor{blue!25}84 & 136 & 220 & $\cdots$ \\
 14 & 23 & 37 & 60 & 97 & \cellcolor{blue!25}157 & 254 & $\cdots$\\
 17 & 28 & 45 & 73 & 118 & 191 & \cellcolor{blue!25}309 & $\cdots$\\
 $\vdots$ & $\vdots$ & $\vdots$ & $\vdots$ & $\vdots$ & $\vdots$ & $\vdots$ & \color{blue}$\ddots$\\
 

\end{tabular}
\end{center}

\begin{parts}
  \part[10] To begin, prove that the Fibonacci series is countable.
 
    \begin{solution}
    % Write your solution here
    From the definition of Fibonacci sequence, any nth term in the Fibonacci sequence can be found as follows,\\
    
    F(n) = F(n-1)+F(n-2) where n,F(n) $\in$ $\mathbb{N}$\\
    
    Starting from 1 and 2, a Fibonacci sequence be\\ 
    
    F=$\{$1, 2, 3, 5, 8,..., F(n)$\}$\\
    
    Every element in the sequence is a natural number, so we can say that the Fibonacci sequence is a subset of $\mathbb{N}$ (set of natural numbers).\\
    
     i.e. 
     $\forall$ n $\in \mathbb{N}$ [ (F(n) $\in$ F) $\wedge$ (F(n) $\in \mathbb{N}) \rightarrow $  F  $\subseteq \mathbb{N}$ ]\\
    
     Since, F $\subseteq$ $\mathbb{N}$, there is a bijection from Fibonacci sequence F to $\mathbb{N}$, so we can infer that the Fibonacci series is countable.
   
    
  \end{solution}
  \part[15] Consider the Modified Wythoff as any array derived from the original, where each entry of the leading diagonal (marked in blue) of the original 2D-Array is replaced with an integer that does not occur in that row. Prove that the Modified Wythoff Array is countable. 

  \begin{solution}
    % Write your solution here
    Suppose, when change the first diagonal, the first element of the first row (i.e. 1) would get changed.\\
    Since, the second sequence(second row) is dependent on the first element of the first row(i.e. 1st element of Sequence2 = 1st element of sequence1 * Golden Ratio)\\
    The sequence would change but it would be another integer sequence which would be a subset of set of Natural numbers.\\
    So, it would still show bijection to the set of natural numbers.\\
    Hence the modified Wythoff array will remain countable.
    
  \end{solution}
\end{parts}

\end{questions}

\end{document}
